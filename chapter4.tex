%%%%%%%%%%%%%%%%%%%%%%%%%%%%%%%%%%%%%%%%%%%%%%%%%%%%%%%%%%%%%%%%%%% 
%                                                                 %
%                           CHAPTER 4                             %
%                                                                 %
%%%%%%%%%%%%%%%%%%%%%%%%%%%%%%%%%%%%%%%%%%%%%%%%%%%%%%%%%%%%%%%%%%% 
 
\chapter{The tropopause-layer static stability structure of tropical cyclones: Idealized modeling}
\resetfootnote %this command starts footnote numbering with 1 again.

%---------------------------------------------------------------------------------------%
\section{Introduction}
%---------------------------------------------------------------------------------------%

The preceding two chapters highlighted the effect of tropical cyclones on the tropopause and upper-level static stability structure in dropsonde observations.
These observations alone, however, cannot explain the mechanisms that force the observed variability.
Numerical simulations of an axisymmetric hurricane conducted in an idealized framework reproduced the observed variability.
Using these simulations, some physical insight into these mechanisms is obtained and described in the present chapter.

%---------------------------------------------------------------------------------------%
\section{Model Setup}
%---------------------------------------------------------------------------------------%

The numerical simulations were performed using version 19.1 of Cloud Model 1 (CM1) described in \cite{BryanRotunno2009} and available online at [WEBSITE].
The fully-compressible, axisymmertic equations of motion were integrated on an Arakawa C-Grid with 1-km horizontal and 250-m vertical grid spacing using a 3rd-order Klemp-Wilhelmson time-splitting integration scheme with 5th-order spatial differencing.
Sub-grid turbulence was parameterized using the planetary boundary layer scheme described in \cite{BryanRotunno2009}, with prescribed horizontal mixing lengths of 100 m at a surface pressure of 1015 hPa and 1000 m at a surface pressure of 900 hPa. 
The model domain was 3000-km-wide and 35-km-deep with a Rayleigh damping layer applied outside of the 5000-km radius and above the 25-km level to prevent spurious gravity wave reflection at the model boundaries.
Microphysical processes were parameterized every advective time step using the Thompson microphysics scheme \cite{Thompson} and radiative heating tendencies were computed every two minutes using the RRTMG [CITATION] longwave and shortwave schemes. 
Temperature and humidity were initialized with a horizontally-homogeneous sounding computed by averaging all dropsondes deployed during the TCI flight conducted on 21 October, 2015 within and around Tropical Storm Patricia (see \cite{DoyleTCI} for details.)
Since relative humidity measurements were unreliable at temperatures below -40 deg C \cite{BellTCI}, the water vapor mixing ratio was assumed to be zero above 16 km.

 
As in the preceding chapters, static stability is analyzed using N2, which is output directly by the model. Potential temperature tendencies 

%---------------------------------------------------------------------------------------%
\section{Results}
%---------------------------------------------------------------------------------------%

The modeled storm's intensity evolution during the first four days of the simulation is depicted in Fig. 1 (blue lines).
Following a 24-hour initial spin-up period, the storm rapidly intensifies, and its maximum 10-m wind speed reaches XXX kt at XXX hours and its minimum central pressure reaches XXX hPa at XXX hours.
This intensification closely resembles that recorded in the NHC best track for Hurricane Patricia (Fig. 1, red stars), indicating that the CM1 simulation is a good representation of Patricia's evolution.
The cold-point tropopause and static stability evolution during this time period is depicted in Fig. 2 for the modeled storm.
The mean tropopause height during the first 24 hours (Day 1; Fig. 2a) lies near XXX km and exhibits little radial variability, with a static stability maximum immediately overlying it in the XXX-XXX-km layer.
As the storm begins its rapid intensification period (Day 2; Fig. 2b), the mean static stability just above the tropopause decreases.
This decrease is particularly pronounced near the storm center, which allows the tropopause to rise at the innermost radii.
The static stability continues to weaken at small radii into Days 3 and 4 (Figs. 2c,d), and the tropopause rises dramatically above the storm's eye.
Meanwhile, outside of the eye region, static stability just above the tropopause strengthens.
This static stability evolution closely follows that observed by the TCI dropsondes deployed in Hurricane Patricia (2015; \cite{DuranMolinari2018}).

%You can add glossary terms like \gls{eregex} and acronyms like \glsname{SRE}.
You can use footnotes.\footnote{Here is a footnote.}
%Don't forget to cite books~\cite{Sipser} and papers~\cite{CarleNarendran}.
