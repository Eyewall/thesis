%%%%%%%%%%%%%%%%%%%%%%%%%%%%%%%%%%%%%%%%%%%%%%%%%%%%%%%%%%%%%%%%%%% 
%                                                                 %
%                            CHAPTER                              %
%                                                                 %
%%%%%%%%%%%%%%%%%%%%%%%%%%%%%%%%%%%%%%%%%%%%%%%%%%%%%%%%%%%%%%%%%%% 
 
\chapter{The tropopause-layer static stability structure of tropical cyclones: Idealized modeling}
\resetfootnote %this command starts footnote numbering with 1 again.

%---------------------------------------------------------------------------------------%
\section{Introduction}
%---------------------------------------------------------------------------------------%

This is a test of git.

The preceding two chapters highlighted the effect of tropical cyclones on the tropopause and upper-level static stability structure in dropsonde observations.
These observations alone, however, cannot explain the mechanisms that force the observed variability.
Numerical simulations of an axisymmetric hurricane conducted in an idealized framework reproduced the observed variability.
Using these simulations, some physical insight into these mechanisms is obtained and described in the present chapter.

%---------------------------------------------------------------------------------------%
\section{Model Setup}
%---------------------------------------------------------------------------------------%

The numerical simulations were performed using version 19 of Cloud Model 1 (CM1) described in \cite{BryanRotunno} and available online at [WEBSITE].
The fully-compressible, axisymmertic equations of motion were integrated on an Arakawa C-Grid with 1-km horizontal and 250-m vertical grid spacing using a 3rd-order Runge-Kutta split-time integration scheme and 5th-order spatial differencing.
Sub-grid turbulence was parameterized using a Smagorinsky scheme with prescribed horizontal and vertical mixing lengths of 1 km and 100 m, respectively.
The model domain was 6000-km wide and 35-km deep with a Rayleigh damping layer applied to horizontal and vertical momentum outside of the 5000-km radius and above the 25-km level to prevent spurious gravity wave reflection at the model boundaries.
Microphysical processes were parameterized every advective time step using the Thompson microphysics scheme \cite{Thompson} and radiative heating tendencies were computed every two minutes using the RRTMG [CITATION] longwave and shortwave schemes. 
 
As in the preceding chapters, static stability is analyzed using N^2, which is output directly by the model. Potential temperature tendencies 

%You can add glossary terms like \gls{eregex} and acronyms like \glsname{SRE}.
You can use footnotes.\footnote{Here is a footnote.}
%Don't forget to cite books~\cite{Sipser} and papers~\cite{CarleNarendran}.
