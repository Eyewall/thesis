%%%%%%%%%%%%%%%%%%%%%%%%%%%%%%%%%%%%%%%%%%%%%%%%%%%%%%%%%%%%%%%%%%% 
%                                                                 %
%                           CHAPTER 5                             %
%                                                                 %
%%%%%%%%%%%%%%%%%%%%%%%%%%%%%%%%%%%%%%%%%%%%%%%%%%%%%%%%%%%%%%%%%%% 
 
\chapter{Conclusions and future work}
\label{chapter:conclusions}
\resetfootnote %this command starts footnote numbering with 1 again.

%---------------------------------------------------------------------------------------%
\section{Summary}
%---------------------------------------------------------------------------------------%

A large rawinsonde dataset collected during the 1998-2011 Atlantic hurricane seasons provided new insight into the tropopause-layer turbulence structure of TCs.
In addition, two new dropsonde datasets collected during HS3 and TCI provided observations of the TC tropopause layer with unprecedented spatiotemporal resolution.
These observations both confirm the results of many previous observational studies and provide new insight into the static stability structure of TCs.
Finally, idealized simulations were able to reproduce the observed structure and evolution of turbulence and static stability in the tropopause layer.
These simulations provided further insight into the physical mechanisms responsible for the observed variability.

\subsection{Upper-tropospheric turbulence}
A large rawinsonde dataset collected during the 1998-2011 Atlantic hurricane seasons provided new insight into the tropopause-layer turbulence structure of TCs.
In particular, environments conducive to turbulence are most common close to the storm center, but can be found all the way out at least the 1000-km radius.
Near the storm center, the 13--14-km layer is most conducive to turbulence, but at outer radii, the most favorable region for turbulence lies near 12 km.

Low-Richardson number layers are considerably more common in hurricanes than in tropical depressions and tropical storms.
The sensitivity to intensity is attributed to the presence of smaller mean upper-tropospheric static stability in hurricanes, with some contribution from larger mean vertical wind shear --- particularly near the tropopause.
This destabilization within hurricanes maximizes just below the tropopause.
These results are supported by idealized simulations of a rapidly-intensifying hurricane, in which the maximum decrease in static stability occurred at and just below the tropopause as the storm intensified.
The larger mean vertical wind shear observed in hurricanes, particularly near the tropopasue, also is observed in these simulations.
Large increases in eddy diffusivity near the tropopause during the storm's intensification are consistent with this layer becoming more conducive to turbulence as static stability decreases and vertical wind shear increases.

A static stability budget analysis reveals that turbulence plays a fundamental role in setting the tropopause-layer potential temperature stratification during TC intensification.
In particular, turbulence destabilizes the layer immediately surrounding the tropopause and stabilizes the lower stratosphere above the tropopause layer.
These results support the outflow layer self-stratification process put forth by \cite{EmanuelRotunno2011}.

\subsection{Tropopause variability in rapidly-intensifying TCs}

Dropsonde observations collected by TCI permitted a high-resolution analysis of the tropopause variability within a rapidly-intensifying TC.
%This analysis revealed a large increase in the height of the cold-point tropopause over the eye of Hurricane Patricia (2015) as it rapidly intensified.
The tropopause height increased dramatically over the eye of Hurricane Patricia (2015) as it rapidly intensified.
This tropopause rise was related to a complete elimination of the tropopause inversion layer (TIL) over the eye as the storm strengthened.

Early during Patricia's RI period, the TIL weakened over a broad region of the storm's inner core.
This weakening was tied to large decreases in lower-stratospheric potential temperature and increases in upper-tropospheric potential temperature.
These potential temperaturetendencies maximized in the layer immediately surrounding the tropopause --- a structure consistent with the effects of turbulent mixing.
It was hypothesized that turbulence, together with detrainment from widespread overshooting convection, forced the lower-stratospheric cooling that helped to weaken the TIL.
This hypothesis is supported by the observations and budget analysis summarized in the previous section.

In the latter stage of Patricia's RI, the inner-core tropopause layer warmed considerably.
This warming, heavily concentrated near the storm center below the tropopause, completed the elimination of the TIL.
As a consequence, the cold-point tropopause rose dramatically and the tropopause temperature warmed over the eye.
Static stability budgets confirmed that vertical gradients of subsidence warming are likely responsible for the elimination of the TIL over the eye.

Outside of Patricia's eye, meanwhile, the TIL restrengthened and the tropopause remained near its initial level.
This re-strengthening also manifested in the idealized CM1 simulations as an increase in static stability during the modeled storm's RI.
Budget analyses revealed that vertical gradients of turbulence and vertical gradients of radiative heating both contributed to stabilizing the lower stratosphere during this period.
It cannot be determined with certainty from the TCI observations whether these processes were primarily responsible for the observed variability in Patricia, but the observations are at least consistent with these mechanisms.

\subsection{The shape of the tropopause and upper-tropospheric static stability in TCs}

Dropsonde observations collected within Hurricane Nadine (2012) during HS3 revealed that the lower-stratospheric stable layer was stronger within the storm's cirrus canopy than outside of the cirrus canopy.
There also existed a distinct static stability maximum below the tropopause within the cirrus canopy that did not exist outside of the cirrus canopy.
It is hypothesized that vertical gradients of radiative heating, vertical gradients of turbulence, and differential advection all could contribute to the formation of this secondary stable layer.

A subjective analysis of all dropsondes released within 500 km of TCs during HS3 suggested that multiple stable layers were most common within regions of cold cirrus clouds.
Likewise, the tropopause was more "sharp" (i.e. the vertical temperature gradient changed more rapidly with height) within regions of cold cirrus.
These results were generalized using the large rawinsonde dataset of \cite{DuranMolinari2016}.
These observations show that the tropopause is colder and sharper in regions of cold cirrus clouds, and the inversion layer at and just above the tropopause is stronger.
In addition, the lower stratosphere is consistently about 3\textdegree{}C colder within regions of cold cirrus than in cloud-free or warmer-cloud regions.
This suggests that deep convection within TCs affects not only the troposphere and tropopause, but also the lower stratosphere up to at least 3 km above the tropopause.

\section{Future work}

The upper troposphere   
