%%%%%%%%%%%%%%%%%%%%%%%%%%%%%%%%%%%%%%%%%%%%%%%%%%%%%%%%%%%%%%%%%%% 
%                                                                 %
%                           CHAPTER 5                             %
%                                                                 %
%%%%%%%%%%%%%%%%%%%%%%%%%%%%%%%%%%%%%%%%%%%%%%%%%%%%%%%%%%%%%%%%%%% 
 
\chapter{Conclusions and future work}
\label{chapter:conclusions}
\resetfootnote %this command starts footnote numbering with 1 again.

%---------------------------------------------------------------------------------------%
\section{Summary}
%---------------------------------------------------------------------------------------%

A large rawinsonde dataset collected during the 1998--2011 Atlantic hurricane seasons provided new insight into the tropopause-layer turbulence structure of TCs.
In addition, two new dropsonde datasets collected during HS3 and TCI provided observations of the TC tropopause layer with unprecedented spatiotemporal resolution.
These observations both confirm the results of many previous observational studies and provide analyses of the static stability structure of TCs with unprecedented resolution.
Finally, idealized simulations were able to reproduce the observed structure and evolution of turbulence and static stability in the tropopause layer.
These simulations provided further insight into the physical mechanisms responsible for the observed variability.

\subsection{Upper-tropospheric turbulence}
A large rawinsonde dataset collected during the 1998--2011 Atlantic hurricane seasons provided new insight into the tropopause-layer turbulence structure of TCs.
In particular, environments conducive to turbulence are most common close to the storm center, but can be found all the way out at least the 1000-km radius.
Near the storm center, the 13--14-km layer is most conducive to turbulence, but at outer radii, the most favorable region for turbulence lies near 12 km.

Low-Richardson number layers are considerably more common in hurricanes than in tropical depressions and tropical storms.
The sensitivity to intensity is attributed to the presence of smaller mean upper-tropospheric static stability in hurricanes, with some contribution from larger mean vertical wind shear --- particularly near the tropopause.
This destabilization within hurricanes maximizes just below the tropopause.
These results are supported by idealized simulations of a rapidly-intensifying hurricane, in which the maximum decrease in static stability occurred at and just below the tropopause as the storm intensified.
The larger mean vertical wind shear observed in hurricanes, particularly near the tropopause, also is observed in these simulations.
Large increases in eddy diffusivity near the tropopause during the storm's intensification are consistent with this layer becoming more conducive to turbulence as static stability decreases and vertical wind shear increases.

A static stability budget analysis reveals that turbulence plays a fundamental role in setting the tropopause-layer potential temperature stratification during TC intensification.
In particular, turbulence destabilizes the layer immediately surrounding the tropopause and stabilizes the lower stratosphere above the tropopause layer.
These results support the outflow layer self-stratification process put forth by \cite{EmanuelRotunno2011}.

\subsection{Tropopause variability in rapidly-intensifying TCs}

Dropsonde observations collected by TCI permitted a high-resolution analysis of the tropopause variability within a rapidly-intensifying TC.
%This analysis revealed a large increase in the height of the cold-point tropopause over the eye of Hurricane Patricia (2015) as it rapidly intensified.
The tropopause height increased dramatically over the eye of Hurricane Patricia (2015) as it rapidly intensified.
This tropopause rise was related to a complete elimination of the tropopause inversion layer (TIL) over the eye as the storm strengthened.

Early during Patricia's RI period, the TIL weakened over a broad region of the storm's inner core.
This weakening was tied to large decreases in lower-stratospheric potential temperature and increases in upper-tropospheric potential temperature.
These potential temperature tendencies maximized in the layer immediately surrounding the tropopause --- a structure consistent with the effects of turbulent mixing.
It was hypothesized that turbulence, together with detrainment from widespread overshooting convection, forced the lower-stratospheric cooling that helped to weaken the TIL.
This hypothesis is supported by the observations and budget analysis summarized in the previous section.

In the latter stage of Patricia's RI, the inner-core tropopause layer warmed considerably.
This warming, heavily concentrated near the storm center below the tropopause, completed the elimination of the TIL.
As a consequence, the cold-point tropopause rose dramatically and the tropopause temperature warmed over the eye.
Static stability budgets confirmed that vertical gradients of subsidence warming are likely responsible for the elimination of the TIL over the eye.

Outside of Patricia's eye, meanwhile, the TIL restrengthened and the tropopause remained near its initial level.
This re-strengthening also manifested in the idealized CM1 simulations as an increase in static stability during the modeled storm's RI.
Budget analyses revealed that vertical gradients of turbulence and vertical gradients of radiative heating both contributed to stabilizing the lower stratosphere during this period.
It cannot be determined with certainty from the TCI observations whether these processes were primarily responsible for the observed variability in Patricia, but the observations are at least consistent with these mechanisms.

\subsection{The shape of the tropopause and upper-tropospheric static stability in TCs}

Dropsonde observations collected within Hurricane Nadine (2012) during HS3 revealed that the lower-stratospheric stable layer was stronger within the storm's cirrus canopy than outside of the cirrus canopy.
There also existed a distinct static stability maximum below the tropopause within the cirrus canopy that did not exist outside of the cirrus canopy.
It is hypothesized that vertical gradients of radiative heating, vertical gradients of turbulence, and differential advection all could contribute to the formation of this secondary stable layer.

A subjective analysis of all dropsondes released within 500 km of TCs during HS3 suggested that multiple stable layers were most common within regions of cold cirrus clouds.
Likewise, the tropopause was more "sharp" (i.e., the vertical temperature gradient changed more rapidly with height) within regions of cold cirrus.
These results were generalized using the large rawinsonde dataset of \cite{DuranMolinari2016}.
These observations show that the tropopause is colder and sharper in regions of cold cirrus clouds, and the inversion layer at and just above the tropopause is stronger.
In addition, the lower stratosphere is consistently about 3\textdegree{}C colder within regions of cold cirrus than in cloud-free or warmer-cloud regions.
This suggests that deep convection within TCs affects not only the troposphere and tropopause, but also the lower stratosphere up to at least 3 km above the tropopause.

\section{Future work}

Much remains to be discovered about the structure and evolution of the TC tropopause layer.
The new observations collected during HS3 and TCI have just begun to be utilized, and there are a number of applications in which these observations could improve our understanding of TC dynamics.

\subsection{The association between the presence of cloud layers and static stability maxima}

It was seen in Chapter \ref{chapter:hs3} that it is more common for multiple upper-tropospheric static stability maxima to be present within regions of cold cirrus clouds.
The relative locations of these stable layers and cloud layers, however, remains unknown.
The Cloud Physics Lidar (CPL) dataset from HS3 provides a unique opportunity to examine the relationship between stability maxima and the presence of cloud layers.
The CPL provides observations of cloud and aerosol optical depth, along with particle characteristics, at 200-m horizontal and 30-m vertical resolution \citep{CPL}.
This allows for the unambiguous detection of cloud layers --- even subvisible cirrus clouds that are not well-observed by satellite imagers \citep{Braunetal2016}, but can exert a strong influence on radiative heating tendencies \citep{Durranetal2009}.
Since CPL flew aboard the same aircraft from which the HS3 dropsondes were deployed on 14--15 September 2012, it is possible to compare directly the temperature profiles observed by the dropsondes to the cloud optical depth profiles observed by CPL in Hurricane Nadine (2012).
If stable layers tend to occur near vertical gradients of optical depth, a relationship can be established definitively between the presence of cirrus clouds and stability maxima.
Even if such a relationship is established, however, it is impossible to know whether the cloud layer is present because the stable layer is present, or whether the stable layer is created by the radiative tendencies associated with the cloud.
Idealized experiments such as those described by \cite{Durranetal2009} and \cite{Dinhetal2010} could provide some insight into how the relationship between cloud layers and the vertical temperature profile evolves.

\subsection{Tropopause-layer static stability and the TC diurnal cycle}

The existence of a diurnal cycle of TC convection has been well established in recent literature (e.g., \citeauthor{Kossin2002} \citeyear{Kossin2002}; \citeauthor{Dunionetal2014} \citeyear{Dunionetal2014}; \citeauthor{BowmanFowler2015} \citeyear{BowmanFowler2015}).
Since this cycle exhibits a convective maximum overnight and in the early morning, and a convective minimum in the afternoon, radiative heating tendencies are a natural suspect in its evolution.
The idealized simulations of \cite{NavarroHakim2016} implicate periodic oscillations in upper-level radiative heating in the evolution of the TC diurnal cycle.
Their results exhibit characteristics of an inertia--gravity wave response with an outward-propagating horizontal phase speed of 9.8 m s\textsuperscript{-1}, which is consistent with the outward motion of the diurnal pulse observed by \cite{Dunionetal2014}. If the diurnal pulse is, indeed, an outward-propagating inertia--gravity wave, the upper-tropospheric static stability profile could have implications for the characteristics of its propagation.
Analysis of high-density dropsonde observations collected through an ongoing diurnal pulse could provide great insight into whether upper-tropospheric static stability might modify its propagation characteristics.

\subsection{Three-dimensional modeling of the TC outflow layer}
Convection and turbulence are fundamentally three-dimensional processes that are not fully captured in the axisymmetric simulations used herein.
In particular, convection manifests in this framework as axisymmetric rings, which is an unrealistic representation of the convective structure of real-world TCs.
High-resolution, three-dimensional simulations are necessary to gain a more physical understanding of the effect of convection and turbulence on the tropopause-layer static stability.
This framework also will permit an analysis of how vertical wind shear and its associated convective asymmetries affect the upper-tropospheric static stability structure.

Finally, the considerable sensitivity of the cloud-top radiative heating tendencies to the vertical grid spacing and radiation parameterization (Chapter \ref{chapter:modeling}) motivates a closer look at the exact nature of this sensitivity.
Comparison of the modeled radiative heating tendencies to observations like those in \cite{Garrettetal2005} could inform the selection of vertical grid spacing and radiation parameterization.
Additionally, more observations of the microphysical properties of the TC cirrus canopy like those in \cite{Heymsfield2006} could help to improve the representation of the TC tropopause layer in numerical models.
Given the importance of cirrus-canopy radiative tendencies shown herein and in previous work (e.g. \citeauthor{Fovelletal2010} \citeyear{Fovelletal2010}; \citeauthor{Buetal2014} \citeyear{Buetal2014}; \citeauthor{Fovelletal2016} \citeyear{Fovelletal2016}), an improved representation of these processes could lead to better forecasts of both TC track and structure.
