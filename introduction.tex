%%%%%%%%%%%%%%%%%%%%%%%%%%%%%%%%%%%%%%%%%%%%%%%%%%%%%%%%%%%%%%%%%%% 
%                                                                 %
%                            INTRODUCTION                         %
%                                                                 %
%%%%%%%%%%%%%%%%%%%%%%%%%%%%%%%%%%%%%%%%%%%%%%%%%%%%%%%%%%%%%%%%%%% 
% \graphicspath{ {figures/introduction/} } %directory for figures
\chapter{Introduction}
\label{intro}

\indent \indent Tropopause temperature is an important parameter in theoretical models of TC structure and intensity.
\cite{EmanuelRotunno2011} expressed the maximum gradient wind speed at the top of the TC boundary layer as a function of upper-level outflow temperature.
They argued that the outflow-layer stratification is determined by a requirement that the Richardson number remains near a critical value for turbulence.
\cite{Emanuel2012} further showed that vortex amplification is a function of the radial gradient of outflow temperature, which is determined by small-scale turbulent mixing.
Thus, within this theoretical framework, upper-tropospheric thermodynamics and turbulence play fundamental roles in determining vortex structure and evolution.
From a climate perspective, changes in tropopause temperature could modify the theoretical maximum intensity that a TC can attain.
\cite{Emanueletal2013} noted that recent increases in potential intensity [as defined by \cite{BisterEmanuel1998}] observed in the North Atlantic region can be explained by observed decreases in temperature near the tropopause.
These theoretical arguments are supported by the three-dimensional radiative-convective equilibrium simulations of \cite{Wangetal2014}, in which the maximum wind speed of TCs increased by 0.4 m s\textsuperscript{-1} per 1 K of tropopause cooling.

Despite its potential importance, most analyses of TC tropopause structure are decades old.
\cite{JordanJordan1954} composited radiosonde ascents through many storms, finding that the tropopause near the storm center was both higher and colder than the surrounding region.
In a case study of four landfalling hurricanes using radiosonde composites, \cite{Koteswaram1967} likewise found an elevated and anomalously cold tropopause at innermost radii in three of the storms.
The one storm that did not have an anomalously cold tropopause [Arlene (1963)] was weakening as it recurved northward.
\cite{Koteswaram1967} argued that the upper-level cold anomaly in the three storms was produced by convection overshooting its level of neutral buoyancy.
These radiosonde studies produced analyses of high vertical resolution, but with the drawback of low horizontal resolution.
This precluded a detailed analysis of finescale horizontal variability in the upper levels of TCs.

A number of aircraft reconnaissance flights in the 1960s observed very strong horizontal temperature gradients in the inner core of hurricanes.
\cite{Penn1966} noted that the tropopause over Hurricane Isbell (1964) sloped upward toward the storm’s inner core by 1.1 km over a 231.5-km horizontal distance.
This elevated tropopause was associated with horizontal temperature gradients as large as 5\textdegree{}C (18.5 km)\textsuperscript{-1} in the lower stratosphere.
The coldest temperatures, -85\textdegree{}C, were observed in the regions of most intense convection.
Likewise in Hurricane Beulah (1967), the coldest temperature, -86\textdegree{}C, was observed a few hundred meters above the highest cloud tops \citep{Waco1970}.
Very near the storm center, the temperature at 16.5-km altitude increased from -86\textdegree{}C to -77\textdegree{}C over a horizontal distance less than 30 km as the aircraft approached the storm center [see Fig. 2 ‘‘Run 2’’ in \cite{Waco1970}].
This strong inward temperature increase was likely associated with an intense upper-tropospheric warm core within and near Beulah’s eye.

The presence of a warm core in the upper troposphere of hurricanes was documented by a series of studies (\citeauthor{LaSeurHawkins1963} \citeyear{LaSeurandHawkins1963}; \citeauthor{HawkinsRubsam1968} \citeyear{HawkinsRubsam1968}; \citeauthor{HawkinsImbembo1976} \citeyear{HawkinsImbembo1976}) using aircraft observations.
\citeauthor{HawkinsImbembo1976} (\citeyear{HawkinsImbembo1976}, their Fig. 6) depicted two anomalous warming maxima---one centered near 600 mb (1 mb = 1 hPa) and another centered near 300 mb---in Hurricane Inez (1966).
The precise location of these warm cores cannot be known with certainty, however, since data were collected at only four levels: 750, 650, 500, and 180 mb.
The upper-level warm core was recognized at least as early as \cite{Haurwitz1935}, who attributed its formation to subsidence warming.
Although a number of authors (e.g., \citeauthor{Malkus1958} \citeyear{Malkus1958}; \citeauthor{Willoughby1979} \citeyear{Willoughby1979}; \citeauthor{Smith1980} \citeyear{Smith1980}) have described this subsidence in theoretical models, its precise effect on warm-core structure is still not fully understood.

Until recently, it was widely accepted that the midlevel warm anomaly observed by \cite{HawkinsImbembo1976} was a departure from the typical TC warm-core structure, and that TCs are typically characterized by an upper-level warm core.
Idealized simulations recently conducted by \cite{SternNolan2012}, however, challenged this perspective.
Although an upper-level perturbation temperature maximum was observed in many of their simulations, all of them exhibited a midlevel temperature anomaly of higher magnitude.
Their results were consistent with \cite{Halversonetal2006}, who observed with dropsondes a maximum temperature anomaly in the midlevels of Hurricane Erin (2001).
Potential temperature ($\theta$) budgets computed by \cite{SternZhang2013} stressed the importance of the vertical profile of static stability in determining the warm core’s precise structure.
Although mean descent within the eye maximized in the 12--13-km layer, subsidence warming was not large there because the static stability was small.
Rather, the maximum temperature anomaly developed in the midtroposphere, where static stability reached a local maximum.
A secondary warming maximum existed near the tropopause, where weaker subsidence coincided with larger static stability in the tropical tropopause layer.
These results are consistent with \cite{OhnoSatoh2015}, whose idealized simulations exhibited a dramatic increase in upper-tropospheric $\theta$ toward the end of a TC’s intensification.
Sawyer--Eliassen diagnostics \cite{PendergrassWilloughby2009} revealed that the contribution of balanced dynamics to the upper-tropospheric warming was dominated by the response to latent heating within the eyewall.
This response was more pronounced later in the period when the vortex grew upward and inertial stability increased in the lower stratosphere, where static stability was large.
\cite{ZhangChen2012} likewise argued that increasing inertial stability concentrated downdrafts in the highly stable lower stratosphere, leading to strong adiabatic warming in this layer.
In numerical simulations, the upper-tropospheric subsidence appears to be related to a narrow inflow layer in the lower stratosphere.
\cite{ChenZhang2013} and \cite{ChenGopalakrishnan2015} assert the importance of the inward advection of higher lower-stratospheric $\theta$ values from outer radii by this inflow layer in developing the upper-tropospheric warm core.
\cite{Kieuetal2016}, on the other hand, did not find evidence for inward $\theta$ advection; rather, they stressed only the importance of subsidence warming as the inflow layer turned downward in the inner core (their Fig. 11a).

Although none of these authors explicitly analyzed the inner-core tropopause variability, the simulations of \cite{OhnoSatoh2015} did show decreased static stability in the lower stratosphere over the eye after the upper-level warm core developed (their Figs. 9c and 10c).
Meanwhile outside of the eye, the static stability just above the tropopause increased as the storm intensified.
The development of this shallow layer of large static stability is consistent with a few early observations of strong, vertically confined temperature inversions just above the tropopause.
An aircraft in Hurricane Beulah (1967) observed a 7\textdegree{}C temperature increase in a vertical layer less than 100 m deep just above the cloud top east of the eye \citep{Waco1970}.
Likewise in Hurricane Isbell (1964), the temperature increased by about 8\textdegree{}C in a layer shallower than 300 m [see Fig. 5 in \cite{Gentry1967}].
It is unclear whether these strong inversions were a consequence of the hurricane’s presence or if the inversions were a part of the background environment.

More recent literature (e.g., \cite{Wirth2003}) has noted that strong, shallow temperature inversions immediately above the cold-point tropopause are a common feature in the tropics, now known as the tropopause inversion layer (TIL).
On the planetary scale, TIL formation and maintenance has been tied to planetary wave dynamics \cite{Griseetal2010} and vertical gradients of radiative heating across the tropopause \citep{Randeletal2007}, but the relative contributions of dynamics and thermodynamics remains uncertain \citep{Ferreiraetal2016}.
To our knowledge, no paper has examined how hurricanes affect the TIL, although the simulations of \cite{OhnoSatoh2015} provide some evidence that they can considerably alter the static stability near the tropopause.
The importance of tropopause temperature in theoretical models of TCs, combined with the potential role of static stability in determining the precise structure of the warm core, motivates a closer look at the evolution of the tropopause and static stability in a hurricane.
This dissertation will employ dropsonde and rawinsonde observations collected within TCs along with idealized modeling to investigate the tropopause-layer static stability structure and evolution of TCs.

%-----Figures-----
\clearpage

%\begin{figure}[h]
  %  \centering
    %\includegraphics{figure_name}
  %  \caption{Caption}
    %\label{label_to_refernce}
%\end{figure}

