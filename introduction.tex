%%%%%%%%%%%%%%%%%%%%%%%%%%%%%%%%%%%%%%%%%%%%%%%%%%%%%%%%%%%%%%%%%%% 
%                                                                 %
%                            INTRODUCTION                         %
%                                                                 %
%%%%%%%%%%%%%%%%%%%%%%%%%%%%%%%%%%%%%%%%%%%%%%%%%%%%%%%%%%%%%%%%%%% 
% \graphicspath{ {figures/introduction/} } %directory for figures
\chapter{Introduction}
\label{intro}

\section{Motivation}

\indent The latter decades of the 20\textsuperscript{th} century saw a remarkable increase in the accuracy of tropical cyclone (TC) track forecasts \citep{PowellAberson2001}, a trend that has continued to the present day.
This increase in skill, largely attributable to improved global numerical weather prediction models and data assimilation schemes \citep{Galletal2013}, has far outpaced intensity forecasts.

TC intensification is fundamentally driven by meso- and micro-scale processes that are not well represented in global models or even regional convection-permitting models \citep{Rogersetal2006}.
Until very recently, many of these processes---particularly those in the boundary layer \citep{Blacketal2007} and upper troposphere \citep{DoyleTCI}---have not been well observed.
However, two recent field campaigns---the National Aeronautics and Space Administration (NASA) Hurricane and Severe Storm Sentinel (HS3; \citeauthor{Braunetal2016} \citeyear{Braunetal2016}) and the Office of Naval Research Tropical Cyclone Intensity (TCI) Experiment \citep{DoyleTCI}---have greatly expanded the number of observations available in the upper troposphere of TCs.
This dissertation will use these observations alongside a large rawinsonde dataset to document the tropopause-layer turbulence and static stability structure of TCs, and how this structure evolves as a TC intensifies.
Idealized modeling also will be employed to explore the mechanisms that might drive the observed evolution.
A better understanding of these processes will add to the overall knowledge of the TC intensification process and hopefully help to improve theoretical and numerical modeling representations of TCs.

\section{The theoretical importance of the tropopause and upper-level turbulence in TCs}
 
Tropopause temperature is an important parameter in theoretical models of TC structure and intensity.
\cite{EmanuelRotunno2011} expressed the maximum gradient wind speed at the top of the TC boundary layer as a function of upper-level outflow temperature.
They argued that the outflow-layer stratification is determined by a requirement that the Richardson number remains near a critical value for turbulence.
\cite{Emanuel2012} further showed that vortex amplification is a function of the radial gradient of outflow temperature, which is determined by small-scale turbulent mixing.
Thus, within this theoretical framework, upper-tropospheric thermodynamics and turbulence play fundamental roles in determining vortex structure and evolution.

From a climate perspective, changes in tropopause temperature could modify the theoretical maximum intensity that a TC can attain.
\cite{Emanueletal2013} noted that recent increases in potential intensity [as defined by \cite{BisterEmanuel1998}] observed in the North Atlantic region can be explained by observed decreases in temperature near the tropopause.
These theoretical arguments are supported by the three-dimensional radiative-convective equilibrium simulations of \cite{Wangetal2014}, in which the maximum wind speed of TCs increased by 0.4 m s\textsuperscript{-1} per 1 K of tropopause cooling.

\section{Definition of the tropopause}

The tropopause can be defined in many ways, depending on the application and the region of interest.

In the midlatitudes, the tropopause often is defined as a surface of constant potential vorticity, typically in the range of 1.5--4 PVU (1 PVU = 10\textsuperscript{-6} K kg\textsuperscript{-1} m\textsuperscript{2} s\textsuperscript{-1}; \citeauthor{Wilcoxetal2011} \citeyear{Wilcoxetal2011}; \citeauthor{HoltonHakim} \citeyear{HoltonHakim}).
This definition requires knowledge of the two-dimensional wind field, however, and cannot be determined using only individual dropsondes or rawinsondes.

The "thermal tropopause" or "lapse-rate tropopause" is defined as "the lowest level at which the lapse rate decreases to 2\textdegree{}C km\textsuperscript{-1} or less, provided that the average lapse rate between this level and all higher levels within 2 km does not exceed 2\textdegree{}C km\textsuperscript{-1}" \citep{WMO1957}.
Since this definition considers only the vertical temperature gradient, the tropopause height and temperature can be determined using a single sounding.
Because the thermal tropopause should be located at a significant level\footnote{A "significant level" is defined as a level "for which values of pressure, temperature, and humidity are reported because temperature and/or moisture-content data at that level are sufficiently important or unusual to warrant the attention of the forecaster, or they are required for the reasonably accurate reproduction of the radiosonde observation." \citep{AMSglossary}.}, this definition is well-suited for low-resolution soundings in which only mandatory and significant levels are reported \citep{YuchechenandCanziani}.
Although widely used for operational applications, the thermal tropopause does not always have physical significance \citep{HighwoodHoskins1998}, and can be erroneous if multiple stable layers exist in the sounding \citep{Hoerlingetal1991}.

The "cold-point tropopause" is defined simply as the level of minimum temperature in a sounding \citep{HighwoodHoskins1998}.
Since this definition requires the detection of a minimum, the sounding used to derive it must have high vertical resolution.
The tropopause so defined has great physical relevance in the tropics.
In particular, the cold-point temperature influences the exchange of ozone and water vapor between the troposphere and stratosphere \citep{Moteetal1996}, which has important climatological implications \citep{Holtonetal1995}.
Although few papers have analyzed the effect of TCs on the tropopause, radar observations suggest that TCs can enhance troposphere--stratosphere exchange through deep convection and turbulent mixing \citep{Dasetal2008}.
In addition, \cite{Davisetal2014} noted that convection within intensifying tropical disturbances can penetrate up to or above the cold-point tropopause, acting to decrease its temperature.
Due to its physical relevance and the availability of high-resolution rawinsonde and dropsonde data to define it, the cold-point tropopause will be used everywhere in this dissertation except in Chapter 2, where the tropopause is not the focus of the results.

\section{Observations of the tropopause in TCs}
Despite its potential importance, most analyses of TC tropopause structure are decades old.
\cite{JordanJordan1954} composited radiosonde ascents through many storms, finding that the tropopause near the storm center was both higher and colder than the surrounding region.
In a case study of four landfalling hurricanes using radiosonde composites, \cite{Koteswaram1967} likewise found an elevated and anomalously cold tropopause at innermost radii in three of the storms.
The one storm that did not have an anomalously cold tropopause [Arlene (1963)] was weakening as it recurved northward.
\cite{Koteswaram1967} argued that the upper-level cold anomaly in the three storms was produced by convection overshooting its level of neutral buoyancy, which is consistent with the more recent findings of \cite{Davisetal2014}.
These radiosonde studies produced analyses of high vertical resolution, but with the drawback of low horizontal resolution.
This precluded a detailed analysis of finescale horizontal variability in the upper levels of TCs.

A number of aircraft reconnaissance flights in the 1960s observed very strong horizontal temperature gradients in the inner core of hurricanes.
\cite{Penn1966} noted that the tropopause over Hurricane Isbell (1964) sloped upward toward the storm’s inner core by 1.1 km over a 231.5-km horizontal distance.
This elevated tropopause was associated with horizontal temperature gradients as large as 5\textdegree{}C (18.5 km)\textsuperscript{-1} in the lower stratosphere.
The coldest temperatures, -85\textdegree{}C, were observed in the regions of most intense convection.
Likewise, in Hurricane Beulah (1967), the coldest temperature, -86\textdegree{}C, was observed a few hundred meters above the highest cloud tops \citep{Waco1970}.
Very near the storm center, the temperature at 16.5-km altitude increased from -86\textdegree{}C to -77\textdegree{}C over a horizontal distance less than 30 km as the aircraft approached the storm center [see Fig. 2 ‘‘Run 2’’ in \cite{Waco1970}].
This strong inward temperature increase was likely associated with an intense upper-tropospheric warm core within, and near, Beulah’s eye.
The anomalous warming associated with this warm core could have implications for the tropopause-layer static stability.

\section{The TC warm core and its influence on the tropopause}

The presence of a warm core in the upper troposphere of hurricanes was documented by a series of studies (\citeauthor{LaSeurHawkins1963} \citeyear{LaSeurHawkins1963}; \citeauthor{HawkinsRubsam1968} \citeyear{HawkinsRubsam1968}; \citeauthor{HawkinsImbembo1976} \citeyear{HawkinsImbembo1976}) using aircraft observations.
\citeauthor{HawkinsImbembo1976} (\citeyear{HawkinsImbembo1976}, their Fig. 6) depicted two anomalous warming maxima---one centered near 600 mb (1 mb = 1 hPa) and another centered near 300 mb---in Hurricane Inez (1966).
The precise location of these warm cores cannot be known with certainty, however, since data were collected at only four levels: 750, 650, 500, and 180 mb.
The upper-level warm core was recognized at least as early as \cite{Haurwitz1935}, who attributed its formation to subsidence warming.
Although a number of authors (e.g., \citeauthor{Malkus1958} \citeyear{Malkus1958}; \citeauthor{Willoughby1979} \citeyear{Willoughby1979}; \citeauthor{Smith1980} \citeyear{Smith1980}) have described this subsidence in theoretical models, its precise effect on warm-core structure is still not fully understood.

Until recently, it was widely accepted that the midlevel warm anomaly observed by \cite{HawkinsImbembo1976} was a departure from the typical TC warm-core structure, and that TCs are typically characterized by an upper-level warm core.
Idealized simulations recently conducted by \cite{SternNolan2012}, however, challenged this perspective.
Although an upper-level perturbation temperature maximum was observed in many of their simulations, all of them exhibited a midlevel temperature anomaly of higher magnitude.
Their results were consistent with \cite{Halversonetal2006}, who observed with dropsondes a maximum temperature anomaly in the midlevels of Hurricane Erin (2001).

Potential temperature ($\theta$) budgets computed by \cite{SternZhang2013} stressed the importance of the vertical profile of static stability in determining the warm core’s precise structure.
Although mean descent within the eye maximized in the 12--13-km layer, subsidence warming was not large there because the static stability was small.
Rather, the maximum temperature anomaly developed in the midtroposphere, where static stability reached a local maximum.
A secondary warming maximum existed near the tropopause, where weaker subsidence coincided with larger static stability in the tropical tropopause layer.
These results are consistent with \cite{OhnoSatoh2015}, whose idealized simulations exhibited a dramatic increase in upper-tropospheric $\theta$ toward the end of a TC’s intensification.
Sawyer--Eliassen diagnostics used by \cite{PendergrassWilloughby2009} revealed that the contribution of balanced dynamics to the upper-tropospheric warming was dominated by the response to latent heating within the eyewall.
This response was more pronounced later in the period when the vortex grew upward and inertial stability increased in the lower stratosphere, where static stability was large.
\cite{ZhangChen2012} likewise argued that increasing inertial stability concentrated downdrafts in the highly stable lower stratosphere, leading to strong adiabatic warming in this layer.

In numerical simulations, upper-tropospheric subsidence appears to be related to a narrow inflow layer in the lower stratosphere.
\cite{ChenZhang2013} and \cite{ChenGopalakrishnan2015} assert the importance of the inward advection of higher lower-stratospheric $\theta$ values from outer radii by this inflow layer in developing the upper-tropospheric warm core.
\cite{Kieuetal2016}, on the other hand, did not find evidence of inward $\theta$ advection; rather, they stressed only the importance of subsidence warming as the inflow layer turned downward in the inner core (their Fig. 11a).

Although none of these authors explicitly analyzed the inner-core tropopause variability, the simulations of \cite{OhnoSatoh2015} did show decreased static stability in the lower stratosphere over the eye after the upper-level warm core developed (their Figs. 9c and 10c).
Meanwhile outside of the eye, the static stability just above the tropopause increased as the storm intensified.
The development of this shallow layer of large static stability is consistent with a few early observations of strong, vertically confined temperature inversions just above the tropopause.
An aircraft in Hurricane Beulah (1967) observed a 7\textdegree{}C temperature increase in a vertical layer less than 100 m deep just above the cloud top east of the eye \citep{Waco1970}.
Likewise in Hurricane Isbell (1964), the temperature increased by about 8\textdegree{}C in a layer shallower than 300 m [see Fig. 5 in \cite{Gentry1967}].
It is unclear whether these strong inversions were a consequence of the hurricane’s presence or if the inversions were a part of the background environment.

More recent literature (e.g., \citeauthor{Wirth2003} \citeyear{Wirth2003}) has noted that strong, shallow temperature inversions immediately above the cold-point tropopause are a common feature in the tropics, now known as the tropopause inversion layer (TIL).
On the planetary scale, TIL formation and maintenance has been tied to planetary wave dynamics \citep{Griseetal2010} and vertical gradients of radiative heating across the tropopause \citep{Randeletal2007}, but the relative contributions of dynamics and thermodynamics remains uncertain \citep{Ferreiraetal2016}.
To our knowledge, no paper has examined how hurricanes affect the TIL, although the simulations of \cite{OhnoSatoh2015} provide some evidence that they can considerably alter the static stability near the tropopause.

\section{The effect of radiative heating near the tropopause}

Recent numerical studies have highlighted the importance of radiative heating in the upper levels of TCs.
\cite{Buetal2014} and \cite{Fovelletal2016} noted that absorption of longwave radiation within the TC cirrus canopy acts to broaden the TC circulation by inducing a deep, radially-extended region of ascent.
The elimination of this within-cloud longwave heating in their simulations produced a more radially-confined cirrus canopy and smaller primary and secondary circulations (see \citeauthor{Buetal2014} \citeyear{Buetal2014}, their Figs. 11,12).
Although the highest-magnitude radiative tendencies in their simulations occurred at the top of the cirrus canopy, where radiative cooling exceeded 0.3 K h\textsuperscript{-1}, these tendencies exerted little impact on the overall storm structure.
This shallow layer of cooling was, however, associated with strong vertical gradients of radiative heating above and below the cloud top.
The magnitude and persistence of this longwave cooling suggests that it could have a direct impact on the static stability near the top of the TC cirrus canopy.
In particular, persistent cooling near the cloud top could act to destabilize the layer below the cloud top and stabilize the layer above.

In addition to the direct effect of radiative cooling on the tropopause-layer static stability, this cooling also could exert an indirect effect by modifying the storm's radial--vertical circulation.
\cite{Durranetal2009} described the circulation that develops in response to radiative heating within tropopause-layer cirrus clouds.
The heating induces upward motion through the heat source, inflow below the heat source, and outflow above.
\cite{Dinhetal2010} showed that these circulations act to spread cirrus clouds laterally, which then would feed back onto the radiative tendencies.
Although the cloud-top radiative cooling played a negligible role in the radiatively-induced storm expansion observed by \cite{Buetal2014} and \cite{Fovelletal2016}, it did modify the circulation near the cloud top.
In particular, it drove weak inflow above the cooling maximum and outflow below, along with subsidence within the region of cooling (see \citeauthor{Fovelletal2016} \citeyear{Fovelletal2016}, Fig. 21).
Although these circulations are weak, their persistence could drive differential advection of potential temperature, as discussed by \cite{ChenZhang2013} and \cite{ChenGopalakrishnan2015},  which could modify the tropopause-layer static stability.

\section{Outline}

The importance of tropopause temperature in theoretical models of TCs, combined with the potential role of static stability in determining the precise structure of the warm core, motivates a closer look at the evolution of the tropopause and static stability in a hurricane.

This dissertation will employ dropsonde and rawinsonde observations, along with idealized modeling, to investigate the structure and evolution of tropopause-layer turbulence and static stability in TCs.
Chapter 2 examines the radial--vertical structure of bulk Richardson number in TCs, how it varies with storm intensity and time of day, and the relative importance of static stability and vertical wind shear in producing layers conductive to turbulence in the upper troposphere.
Chapter 3 analyzes the tropopause and upper-level static stability evolution of Hurricane Patricia (2015) during its rapid intensification, and hypothesizes mechanisms that might have led to the observed variability.
Chapter 4 employs idealized, axisymmetric simulations to reproduce the tropopause and static stability tendencies observed in Hurricane Patricia.
Budget analyses are performed to isolate the most important mechanisms that drive the tropopause variability in the modeled storm.
Chapter 5 presents the upper-level static stability structure of category one Hurricane Nadine (2012) and provides a preliminary analysis of the tropopause characteristics of a large number of TCs.

%-----Figures-----
\clearpage

%\begin{figure}[h]
  %  \centering
    %\includegraphics{figure_name}
  %  \caption{Caption}
    %\label{label_to_refernce}
%\end{figure}

