%%%%%%%%%%%%%%%%%%%%%%%%%%%%%%%%%%%%%%%%%%%%%%%%%%%%%%%%%%%%%%%%%%% 
%                                                                 %
%                         ACKNOWLEDGEMENT                         %
%                                                                 %
%%%%%%%%%%%%%%%%%%%%%%%%%%%%%%%%%%%%%%%%%%%%%%%%%%%%%%%%%%%%%%%%%%% 
 
\specialhead{PREFACE}

%---------------------------------------------------------------------------------------%

\indent \indent This dissertation is based in part upon research previously published by the author in peer-reviewed journals.
The results in Chapter 2 were published in the \textit{Journal of the Atmospheric Sciences} (Duran and Molinari 2016) and those in Chapter 3 were published in \textit{Monthly Weather Review} (Duran and Molinari 2018a).
All results in Chapter 4 except the sensitivity experiments described in Section \ref{sensitivitysection} have been submitted to \textit{Journal of the Atmospheric Sciences} (Duran and Molinari 2018b) and are currently under review.
The introductions to each of these papers have been condensed and integrated into a single, unified introduction (Chapter 1).

\vspace{10mm}

\noindent \textbf{References:}

\noindent Duran, P., and J. Molinari, 2016: Upper-tropospheric low Richardson number in tropical cyclones: Sensitivity to cyclone intensity and the diurnal cycle. \textit{J. Atmos. Sci.}, \textbf{73}, 545-554.

\noindent Duran, P., and J. Molinari, 2018: Dramatic inner-core tropopause variability during the rapid intensification of Hurricane Patricia (2015). \textit{Mon. Wea. Rev.}, \textbf{146}, 119-134.

\noindent Duran, P., and J. Molinari, 2018: Tropopause evolution in a rapidly intensifying tropical cyclone: A static stability budget analysis in an idealized, axisymmetric framework. \textit{J. Atmos. Sci.}, in review.
