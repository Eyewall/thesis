%%%%%%%%%%%%%%%%%%%%%%%%%%%%%%%%%%%%%%%%%%%%%%%%%%%%%%%%%%%%%%%%%%% 
%                                                                 %
%                           CHAPTER 4                             %
%                                                                 %
%%%%%%%%%%%%%%%%%%%%%%%%%%%%%%%%%%%%%%%%%%%%%%%%%%%%%%%%%%%%%%%%%%% 
 
\chapter{Upper-tropospheric low Richardson number in tropical cyclones: Sensitivity to cyclone intensity and the diurnal cycle}
\resetfootnote %this command starts footnote numbering with 1 again.

%---------------------------------------------------------------------------------------%
\section{Introduction}
%---------------------------------------------------------------------------------------%

In addition to introduction in the published paper, tie this in to the Patricia and CM1 results...

%---------------------------------------------------------------------------------------%
\section{Data and methods}
%---------------------------------------------------------------------------------------%
The U.S. High Vertical Resolution Radiosonde Data archive \citep{LoveGeller2012} is used to construct the radial–vertical structure of TC outflow.
All sondes released within 1000 km of Atlantic basin TCs from 1998 to 2011 are used, including special soundings released at 0600 and 1800 UTC.
The dataset includes sondes released from the eastern United States, the Bahamas, Puerto Rico, Grand Cayman, and Belize.
Herein we analyze the vertical layer between 9 and 17 km, which is at the upper limit of---or above---the analyses performed by \cite{Molinarietal2014}.

Storm center positions are determined using the National Hurricane Center Atlantic basin hurricane database (HURDAT; \citeauthor{Jarvinenetal1984} \citeyear{Jarvinenetal1984}).
For the 1998–2011 time interval, HURDAT contained 1807 hurricane, 2500 tropical storm, and 1192 tropical depression analysis times.
Sondes are separated into 200-km radial bins about a composite TC center, with each bin overlapping the adjacent bins by 100 km.
This binning ensures that there are sufficient observations in each bin for azimuthal averaging but does not allow for the TC inner core to be resolved.

Every sounding was visually examined to ensure quality, and 218 questionable soundings---constituting less than 3\% of the total number---were removed.
Of these 218 soundings, 130 were removed because data gaps larger than 1 km arose in the 9--17-km layer; another 84 were removed because they contained upper-tropospheric  vertical  wind shears larger than 40 m s\textsuperscript{-1} km\textsuperscript{-1}; and 3 were removed because they contained multiple observations on the same pressure level.

As in \cite{Molinarietal2014}, the remaining 8499 soundings were interpolated to 100-m vertical levels and smoothed using a 1–2–1 smoother.
The bulk Richardson number

   \begin{equation} \label{eq:rb}
   R_b = \frac{(g/\theta_v)(\Delta\theta_v/\Delta z)}{[(\Delta u)^2+(\Delta v)^2]/(\Delta z)^2},
   \end {equation}
where $\theta_v$ is the layer-averaged virtual potential temperature, was then computed for 400-m ($\Delta z$) layers centered on each level.
The percentage of rawinsondes that observe $R_b < 0.25$ is plotted as a composite radius--height cross section, along with the azimuthally averaged stability and shear contributions to $R_b$ [azimuthal averages of the numerator and denominator of Eq. (1), respectively].
These components of $R_b$ are plotted as difference fields to show variability with TC intensity and time of day, and the statistical significance of these differences is tested using a 10 000 sample bootstrapping technique.
Probability distributions also are used to assess the contribution of transient perturbations to the generation of low-$R_b$ layers.
To avoid double counting, these probability distributions do not use overlapping bins.

To  aid  in  the  diagnosis  of  changes  in  upper-tropospheric structure, average tropopause heights are calculated for each intensity and time stratification.
The tropopause height is determined for each sounding using the following World Meteorological Organization definition: ‘‘the lowest level at which the lapse rate decreases to 2\textdegree{}C km\textsuperscript{-1} or less, provided that the average lapse rate between this level and all higher levels within 2 km does not exceed 2\textdegree{}C km\textsuperscript{-1} \citep{WMO1957}.
The tropopause heights are then averaged using the same radial binning as the $R_b$ fields and overlaid on the stability and shear cross sections.

Flight-level vertical accelerations observed by the G-IV aircraft, recorded at a frequency of 1 Hz in Hurricane Ivan (2004), are used to provide an example of upper-tropospheric turbulence.
All observations used herein were recorded while the aircraft was above the 12-km altitude.
The G-IV’s true airspeed for these data varies between 220 and 242 m s\textsuperscript{-1}, placing the horizontal spacing of observations at around 230 m.

%---------------------------------------------------------------------------------------%
\section{Results}
%---------------------------------------------------------------------------------------%

\subsection{Aircraft turbulence observations}

Given that aircraft often observe turbulence while flying in the outflow of mesoscale convective systems, we first consider turbulence observed on an aircraft in the vicinity of a hurricane.
Vertical acceleration from a G-IV flight into Hurricane Ivan (2004) is shown in Fig. \ref{fig:ir+accel}, along with the flight track overlaid on an infrared satellite image.
Plotting begins at the black asterisk, and changes in the color of the flight track corresponding to changes in the color of the vertical acceleration trace.

The area within the cirrus canopy is more turbulent than the clear air surrounding the storm (Fig. \ref{fig:ir+accel}).
As the aircraft enters the cirrus region, the vertical acceleration exhibits considerably larger variance, and the variance remains elevated until the aircraft exits the cirrus canopy.
As the G-IV turns eastward and repenetrates the cirrus, the vertical acceleration variance again increases and remains elevated until the aircraft once again exits the storm.
The observations here are not collected in the inner core of strong convection, but in high-cloud regions at outer radii.
Thus, the turbulence is not likely to be the direct result of strong mesoscale convection, but instead the consequence of other mechanisms unique to the cirrus canopy and the near-storm environment.
Although the magnitude of the accelerations corresponds to only light turbulence \citep{WMO1998}, such a striking difference between the within-cirrus and clear-air environments suggests that there exist mechanisms unique to the cirrus canopy that produce turbulence.

\subsection{Low Richardson number, stability, and shear in the upper troposphere}

The occurrence of low $R_b$ in the upper troposphere of TCs is illustrated in Fig. \ref{fig:rb-all}, using all soundings in the dataset.
The frequency of $R_b < 0.25$ decreases outward from a maximum of over 9\% at inner radii to 2\%–3\% at 1000 km.
At inner radii, low-$R_b$ frequency is maximized near the 13.5-km level, whereas at outer radii the maximum is near 11.5 km.
A downward radial slope is present throughout the composite along with a gradual thinning of the low-$R_b$ layer with increasing radius.

The static stability and vertical wind shear terms of $R_b$ [the  numerator  and  denominator  of  Eq. \ref{eq:rb},  respectively] for all data points between the 11- and 15-km levels are shown in Fig. 3.
The distributions of static stability (red lines) and vertical wind shear (blue lines) both are maximized near zero and fall off rapidly with increasing values.
The lack of a left tail in the distributions is a consequence of the physical nature of the static stability and vertical wind shear terms: in the absence of strong forcing, a static stability less than zero should be rapidly mixed out, so negative static stabilities are rare, and the vertical wind shear term is defined to be greater than zero.

The static stability distribution for points with $R_b < 0.25$ (solid red line) is shifted toward smaller values relative to the same distribution for data points that observe larger $R_b$ (dotted red line).
This indicates that local decreases in static stability contribute to the generation of low-$R_b$ layers.
Similarly, the vertical wind shear distribution for all data points with $R_b < 0.25$ (solid blue line) is shifted toward larger values relative to the same distribution for data points that observe larger $R_b$ (dotted blue line).
The upper tail of this distribution extends to very large values, indicating that high-magnitude vertical wind shear perturbations also contribute to the generation of low-$R_b$ layers.

These results are consistent with \cite{Molinarietal2014}, who found that the frequency of low $R_b$ is maximized in the upper troposphere, and that both static stability and vertical wind shear perturbations contribute to its production.

%---------------------------------------------------------------------------------------%
\section{Discussion}
\label{sec:discussion}
%---------------------------------------------------------------------------------------%

%---------------------------------------------------------------------------------------%
%FIGURES AND TABLES
%---------------------------------------------------------------------------------------%

%FIGURE 1%
\begin{figure*}[ht]
\centerline{\includegraphics[width=39pc]{figures/fig1_flightlevelcdo.png}}
\caption{(a) The G-IV flight track into Hurricane Ivan between 0530 and 1310 UTC 15 Sep 2004 overlaid on an infrared satellite image valid at 0915 UTC. (b) Time series of aircraft vertical acceleration observed by the G-IV inertial navigation system. The initial time for both plots is indicated by the black asterisk in (a), and the color changes in the flight track correspond to the color changes in (b). All data plotted here were collected while the G-IV was at an altitude of 12 km or greater.}
\label{fig:ir+accel}
\end{figure*}

%FIGURE 2%
\begin{figure*}[ht]
\centerline{\includegraphics[width=39pc]{figures/fig2_brch_allsondes.png}}
\caption{Radius–height cross section of the percentage of rawinsondes that observed a bulk Richardson number less than 0.25 for all available sondes released within 1000 km of an Atlantic basin tropical cyclone. Because vertical resolution is much larger than radial resolution, a 1–2–1 smoother is applied in the vertical 10 times for plotting purposes.}
\label{fig:rb-all}
\end{figure*}

%FIGURE 3%
\begin{figure*}[ht]
\centerline{\includegraphics[width=39pc]{figures/fig3_dist_allsondes.png}}
\caption{ Probability distributions of the static stability and vertical
wind shear terms (10\textsuperscript{-4} s\textsuperscript{-2}) of the bulk Richardson number for all data points between the 11- and 15-km levels. Red curves represent static stability and blue curves represent vertical wind shear. Solid curves are for all points where the bulk Richardson number is smaller than 0.25, and dotted curves are for points where the bulk Richardson number is greater than or equal to 0.25. The small number of apparently negative shear-squared values is an artifact of the use of finite bins in the plotting routine.}
\label{fig:dist-all}
\end{figure*}

%FIGURE 4%
\begin{figure*}[ht]
\centerline{\includegraphics[width=39pc]{figures/fig4_brch_intensity.png}}
\caption{As in Fig. \ref{fig:rb-all}, but for all sondes released within 1000 km of (a) hurricanes and (b) tropical depressions and tropical storms.}
\label{fig:rb-strat}
\end{figure*}

%FIGURE 5%
\begin{figure*}[ht]
\centerline{\includegraphics[width=39pc]{figures/fig5_stabshear_intensity.png}}
\caption{Radius–height cross sections of the differences in azimuthally averaged (a) numerator (stability) and (b) denominator (shear) of Eq. \ref{eq:rb} in hurricanes vs weak TCs. The difference (10\textsuperscript{-4} s\textsuperscript{-2}) is plotted such that blue colors represent values that are smaller in hurricanes than in weak TCs. Because vertical resolution is much larger than radial resolution, a 1–2–1 smoother is applied in the vertical 10 times for plotting purposes. The solid orange line is the average tropopause height for hurricanes, and the dashed orange line is the average for weak TCs. Green stippling indicates regions where the differences are statistically significant at the 99\% confidence interval.}
\label{fig:stabsheardiffs}
\end{figure*}

%FIGURE 6%
\begin{figure*}[ht]
\centerline{\includegraphics[width=39pc]{figures/fig6_temp_diff.png}}
\caption{Radius–height cross section of the azimuthally averaged temperature (K) in hurricanes minus that in tropical depressions and tropical storms, such that blue colors represent lower temperatures in hurricanes. Because vertical resolution is much larger than radial resolution, a 1–2–1 smoother is applied in the vertical 10 times for plotting purposes. The solid orange line is the average tropopause height for hurricanes, and the dashed orange line is the average tropopause height for weak TCs.}
\label{fig:tempdiffs}
\end{figure*}

%FIGURE 7%
\begin{figure*}[ht]
\centerline{\includegraphics[width=39pc]{figures/fig7_pdf_perturbs_intensity.png}}
\caption{Probability distributions of the perturbation static stability and vertical wind shear terms (10\textsuperscript{-4} s\textsuperscript{-2}) of the bulk Richardson number for hurricanes and weak TCs. Perturbations are computed by subtracting the azimuthal average from each individual observation within the 400-km radius and between the 11- and 15-km levels, using 100-km-wide, nonoverlapping radial bins. The physical characteristics of static stability and vertical wind shear cause the perturbation distributions to be primarily negative, as discussed in Section \ref{section:resultssection??}.}
\label{fig:dist-pert}
\end{figure*}

%FIGURE 8%
\begin{figure*}[ht]
\centerline{\includegraphics[width=39pc]{figures/fig8_brch_intensity.png}}
\caption{As in Fig. \ref{fig:rb-strat}, but for all sondes released at (a) 0000 UTC (evening) and (b) 1200 UTC (morning). Sondes released at 0600 and 1800 UTC are not considered here because their numbers are much fewer and the observations are skewed toward more intense TCs. The exclusion of these sondes from more intense TCs yields a smaller overall frequency of $R_b < 0.25$  here than in Fig. \ref{fig:rb-strat}.}
\label{fig:rb-time}
\end{figure*}

%FIGURE 9%
\begin{figure*}[ht]
\centerline{\includegraphics[width=39pc]{figures/fig9_stabshear_diurnal.png}}
\caption{As in Fig. \ref{fig:stabsheardiffs} , but for all sondes released at 1200 UTC (morning) minus all sondes released at 0000 UTC (evening), such that blue colors represent values that are smaller at 1200 UTC than at 0000 UTC. The absence of green stippling indicates that these differences are not statistically significant at the 99\% confidence interval.}
\label{fig:stabsheartime}
\end{figure*}

%FIGURE 10%
\begin{figure*}[ht]
\centerline{\includegraphics[width=39pc]{figures/fig10_pdf_perturbs_diurnal_0-400km_11-15km_brch_-3-3.png}}
\caption{As in Fig. \ref{fig:dist-pert}, but at 0000 UTC (solid lines) and 1200 UTC (dotted lines).}
\label{fig:dist-pert-time}
\end{figure*}
